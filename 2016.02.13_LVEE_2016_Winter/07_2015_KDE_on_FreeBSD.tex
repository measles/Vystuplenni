\documentclass[ignorenonframetext,hyperref={pdftex,unicode}]{beamer}

\usetheme{AndrejZ}

\title{Свой уласны Dropbox, з календаром, кантактамі і RSS}
\author[Андрэй Захарэвіч]{Андрэй Захарэвіч\\ andrej@zahar.ws}


\begin{document}

\frame{\titlepage} 


\section{Мэты і задачы} 

\begin{frame}{Што мы хацелі атрымаць} 
	Для пачатку хацелася б атрымаць незалежны ад пастаўшчыка інструмент для штодзённых воблачных задач, якія звычайна мы давяраем карпарацыям кшталту Dropbox Inc, Google Inc і іншым. Асноўныя задачы, якія я б хацеў пакрыць:\pause
	\begin{itemize}
		\item Воблачнае файлавае сховішча/файлаабменнік
		\item Кантакты і календар
		\item RSS
		\item Спасылкі
	\end{itemize}
	\begin{center}
 		%\includegraphics[height=0.5\textheight,keepaspectratio]{freebsd_badge} %так вставляется картинка
	\end{center}
\end{frame}

\begin{frame}{Файлаабменнік}
	\begin{itemize}
		\item Звычайныя магчымасці па абнаўленню файлаў з адной крыніцы на ўсе месцы, дзе ляжыць яго копія \pause
		\item Магчымасць стварыць агульны набор рэсурсаў для некалькіх карытальнікаў \pause
		\item Магчымасць падзяліцца спасылкай на нейкі файл з кім заўгодна (з нейкім механізмам абмежавання доступа) \pause
		\item Кліент пад Linux для сінхранізацыі ўсяго акаўнта ці асобных яго папак. Пажадана, каб дазваляў дадаваць некалькі акаўнтаў \pause
		\item Кліент пад Android з магчымасцю выбарачнай сінхранізацыі да ўзроўню асобнага файла
	\end{itemize}
\end{frame}

\begin{frame}{Кантакты і календар}
	\begin{itemize}
		\item Цэнтралізаванае сховішча для календара і кантактаў з магчымасцю сінхранізацыі на прылады Android. У ідэале каб сінханізацыя адбывалася гэтак жа, як са звычайнымі правайдэрамі, то бок как не абмяжоўваць спіс праграм, якія потым з гэтымі кантактамі ці календарнымі запісамі змогуць працаваць \pause
		\item Зручны (хаця б мінімальна) інтэрфейс для кампьютэра з магчымасцю аб’ядноўваць кантакты ў групы \pause
		\item Групы кантактаў павінны быць бачныя і даступныя для любых аперацый як з кампа, так і з тэлефона \pause
		\item Магчымасць імпарту кантактаў і календара ў Thunderbird \pause
		\item Магчымасць экспарту у файл і пераносу на іншы сервер
	\end{itemize}
\end{frame}

\begin{frame}{RSS}
	\begin{itemize}
		\item Магчымасць чытаць як з кампьютэра, так і з Android-прылад \pause
		\item Магчымасць дадаваць стужкі з кампьютэра і андроід-прылад \pause
		\item Магчымасць сінхранізаваць стан (прачытаны, пазначаны як цікавы) для розных працоўных месцаў \pause
		\item У ідэале, магчымасць атрымаць доступ праз вэб-інтэрфейс \pause
		\item Магчымасць экспарту у файл і пераносу на іншы сервер
	\end{itemize}
\end{frame}

\begin{frame}{Спасылкі}
	\begin{itemize}
		\item Магчымасць чытаць як з кампьютэра, так і з Android-прылад \pause
		\item Магчымасць дадаваць з кампьютэра і Андроід-прылад \pause
		\item Групіроўка па групах ці тэгах, даступная для ўсіх кліентаў \pause
		\item Магчымасць экспарту у файл і пераносу на іншы сервер
	\end{itemize}
\end{frame}

\frame{\questionslide}

\begin{frame}{Ужытыя малюнкі}
	\begin{thebibliography}{10}
	\beamertemplatetextbibitems
	\bibitem{}
		{\sc \href{http://amai-biscuit.deviantart.com/art/FreeBSD-Badge-345132138}{FreeBSD Badge}} by {\sc \href{http://amai-biscuit.deviantart.com/}{amai-biscuit}};
	\bibitem{}
		{\sc \href{https://www.flickr.com/photos/nmcmanus/338391435}{Easy Button}} by {\sc \href{https://www.flickr.com/photos/nmcmanus/}{Civilian Scrabble}};
	\bibitem{}
		{\sc \href{https://www.flickr.com/photos/martinaphotography/6428406857}{66/365 Another Creepy One}} by {\sc \href{https://www.flickr.com/photos/martinaphotography/}{martinak15}};
	\bibitem{F-16}
		{\sc \href{https://commons.wikimedia.org/wiki/File:F-15\_takeoff.jpg}{F-15\_takeoff}} by {\sc USAF};
	\bibitem{}
		{\sc \href{https://www.flickr.com/photos/toasty/914441359}{Bang!}} by {\sc \href{https://www.flickr.com/photos/toasty/}{Kenneth Lu}}.
	\end{thebibliography}
\end{frame}

\end{document}
